\documentclass[11pt,letterpaper]{article}
\usepackage[top=3cm, bottom=2cm, left=2cm, right=2cm, columnsep=20pt]{geometry}
\usepackage{pdfpages}
\usepackage{graphicx}
\usepackage{etoolbox}
\apptocmd{\sloppy}{\hbadness 10000\relax}{}{}
% \usepackage[numbers]{natbib}
\usepackage[T1]{fontenc}
\usepackage{ragged2e}
\usepackage[french]{babel}
\usepackage{listings}
\usepackage{color}
\usepackage{soul}
\usepackage[utf8]{inputenc}
\usepackage[export]{adjustbox}
\usepackage{caption}
\usepackage{amsmath}
\usepackage{amssymb}
\usepackage{float}
\usepackage{csquotes}
\usepackage{fancyhdr}
\usepackage{wallpaper}
\usepackage{siunitx}
\usepackage[indent]{parskip}
\usepackage{textcomp}
\usepackage{gensymb}
\usepackage{multirow}
\usepackage[hidelinks]{hyperref}
\usepackage{abstract}
\renewcommand{\abstractnamefont}{\normalfont\bfseries}
\renewcommand{\abstracttextfont}{\normalfont\itshape}
\usepackage{titlesec}
\titleformat{\section}{\large\bfseries}{\thesection}{1em}{}
\titleformat{\subsection}{\normalsize\bfseries}{\thesubsection}{1em}{}
\titleformat{\subsubsection}{\normalsize\bfseries}{\thesubsubsection}{1em}{}

\usepackage{xcolor}
\definecolor{codegreen}{rgb}{0,0.6,0}
\definecolor{codegray}{rgb}{0.5,0.5,0.5}
\definecolor{codepurple}{rgb}{0.58,0,0.82}
\definecolor{backcolour}{rgb}{0.95,0.95,0.92}
\lstdefinestyle{mystyle}{
    backgroundcolor=\color{backcolour},   
    commentstyle=\color{codegreen},
    keywordstyle=\color{magenta},
    numberstyle=\tiny\color{codegray},
    stringstyle=\color{codepurple},
    basicstyle=\ttfamily\footnotesize,
    breakatwhitespace=false,         
    breaklines=true,                 
    captionpos=b,                    
    keepspaces=true,                 
    numbers=left,                    
    numbersep=5pt,                  
    showspaces=false,                
    showstringspaces=false,
    showtabs=false,                  
    tabsize=2
}
\lstset{style=mystyle}

\usepackage[most]{tcolorbox}
\newtcolorbox{note}[1][]{
  enhanced jigsaw,
  borderline west={2pt}{0pt}{black},
  sharp corners,
  boxrule=0pt, 
  fonttitle={\large\bfseries},
  coltitle={black},
  title={Note:\ },
  attach title to upper,
  #1
}

%----------------------------------------------------

\setlength{\parindent}{0pt}
\DeclareCaptionLabelFormat{mycaptionlabel}{#1 #2}
\captionsetup[figure]{labelsep=colon}
\captionsetup{labelformat=mycaptionlabel}
\captionsetup[figure]{name={Figure }}
\newcommand{\inlinecode}{\normalfont\texttt}
\usepackage{enumitem}
\setlist[itemize]{label=\textbullet}

\begin{document}
\begin{titlepage}
\center

\begin{figure}
    \ThisULCornerWallPaper{.4}{Polytechnique_signature-RGB-gauche_FR.png}
\end{figure}
\vspace*{2 cm}

\textsc{\Large \textbf{PHS2223 --} Introduction à l'optique moderne}\\[0.5cm]
\large{\textbf{Équipe : 04}}\\[1.5cm]

\rule{\linewidth}{0.5mm} \\[0.5cm]
\Large{\textbf{Expérience 1}} \\[0.2cm]
\text{Microscopie confocale}\\
\rule{\linewidth}{0.2mm} \\[2.3cm]

\large{\textbf{Présenté à}\\
  Guillaume Sheehy\\
  Esmat Zamani\\[2.5cm]
  \textbf{Par :}\\
  Émile \textbf{Guertin-Picard} (2208363)\\
  Laura-Li \textbf{Gilbert} (2204234)\\
  Tom \textbf{Dessauvages} (2133573)\\[3cm]}

\large{\today\\
Département de Génie Physique\\
Polytechnique Montréal\\}

\end{titlepage}

%----------------------------------------------------

\tableofcontents
\pagenumbering{roman}
\newpage

\pagestyle{fancy}
\setlength{\headheight}{14pt}
\renewcommand{\headrulewidth}{0pt}
\fancyfoot[R]{\thepage}

\pagestyle{fancy}
\fancyhf{}
\renewcommand{\headrulewidth}{1pt}
\fancyhead[L]{\textbf{PHS2223}}
\fancyhead[C]{titre}
\fancyhead[R]{date}
\fancyfoot[R]{\thepage}

\pagenumbering{arabic}
\setcounter{page}{1}

%----------------------------------------------------

\section{Résultats}

\subsection{Estimation de la résolution}

Les figures \ref{brutes 1} et \ref{brutes 2} présentent sous la forme de graphiques, les données brutes recueillies lors du laboratoire. 

% Résultats brut 
\begin{figure}[h!]
    \centering
    \begin{minipage}[t]{0.46\linewidth}
        \centering
        \includegraphics[scale=0.32]{Données brutes 1.png}
        \caption{Données brutes obtenues pour le système avec pinhole}
        \label{brutes 1}
    \end{minipage}\hfill
    \begin{minipage}[t]{0.5\linewidth}
        \centering
        \includegraphics[scale=0.32]{Données brutes 2.png}
        \caption{Données brutes obtenues pour le système sans pinhole}
        \label{brutes 2}
    \end{minipage}

\end{figure}

Au regard de l'allure de ces données, la fonction utilisées pour en faire un fit est une somme de fonctions gaussiennes centrées selon les positions des différents maximums d'amplitude. L'utilisation de la librairie \textit{lm.fit} de python permet de déterminer les valeurs optimales de ces courbes de tendances et de les représenter. Les figures \ref{fit 1} et \ref{fit 2} présentent sous la forme de graphiques, les fonctions de fitting obtenues pour les deux jeux de données brutes différents. 

\begin{figure}[h!]
    \centering
    \begin{minipage}[t]{0.46\linewidth}
        \centering
        \includegraphics[scale=0.32]{Données fit 1.png}
        \caption{Partie réelle et imaginaire du signal brut pour 10ms}
        \label{fit 1}
    \end{minipage}\hfill
    \begin{minipage}[t]{0.5\linewidth}
        \centering
        \includegraphics[scale=0.32]{Données fit 2.png}
        \caption{Phase du signal brut pour 10ms}
        \label{fit 2}
    \end{minipage}
\end{figure}

Ces représentations du signal sous la forme de somme de courbes gaussiennes permettent de mettre en évidence deux maximums d'amplitudes pour chacune des séries de données. L'amplitude représentative de la résolution du système est celle de valeur maximale. Cette résolution, représentée comme étant la largeur de la courbe à mi-hauteur de l'amplitude maximale peut, elle aussi être estimée à l'aide de \textit{lm.fit}. Le tableau \ref{FMWH} représente les résolutions obtenues ainsi que les erreurs qui leurs sont associées en fonction de la série de données :

\begin{table}[H]
\centering
\begin{tabular}{|p{3.2cm}|p{4.2cm}|p{3.2cm}|}
\cline{2-3}
\multicolumn{1}{c|}{} & \textbf{Largeur à mi-hauteur} & \textbf{Erreur} \\
\hline
\textbf{Avec pinhole} & 1.317578 & 0.092557 \\
\hline
\textbf{Sans pinhole} & 5.742349 & 0.943043 \\
\hline
\end{tabular}
\caption{Largeur à mi-hauteur et erreurs en fonction du jeu de données}
\label{FMWH}
\end{table}

\subsection{Analyse de données de LSCM en fluorescence}

Une image 3D acquise en microscopie confocale par un microscope \textit{Leica TCS SP5 MP} est analysée. Les données
acquises au préalable sont des cellules de prostate \textit{RWPE-1}, avec plusieurs cellules de cancer de la prostate
\textit{PC3} présentes. La figure \ref{non_yes_confo} présente côte à côte une image de ces cellules pour comparer
l'image d'un système non-confocal à l'image d'un système confocal.

\begin{figure}[H]
  \centering
  \includegraphics[scale=0.45]{non_and_yes_confocal.png}
  \caption{Comparaison de l'imagerie obtenue d'un système non-confocal (à gauche) à celle d'un système confocal (à droite).}
  \label{non_yes_confo}
\end{figure}

Il est possible de voir une meilleure résolution sur l'acquisition confocale, ce qui résulte en des contours plus définis.
Ensuite, la figure \ref{3d_cells} présente trois coupes de l'image 3D autour du même point dants l'espace afin de
représenter le volume de quelques cellules.

\begin{figure}[H]
  \centering
  \includegraphics[scale=0.34]{volume.png}
  \caption{Volume de deux cellules bien visibles par LSCM en fluorescence présenté en trois plans.}
  \label{3d_cells}
\end{figure}


\section{Discussion}

\subsection{Retour sur l'hypothèse}

%TODO : comparaision mesure axiale entre hypothese et expérience

\subsection{Analyse des causes d'erreurs}

%TODO

vieux laser avec deux pics d'intensité
mauvais alignement de base (surtout l'alignement des back reflections)
désalignement au fur et à mesure des manipulations
incertitudes (surtout sur le power meter)

\subsection{Question 1}

%TODO
Le fonctionnement de base de la microscopie confocale par fluorescence repose sur l'utilisation de la fluorescence afin de visualiser des points spécifiques de l'échantillon \textcolor{red}{(Source 1)}. En d'autres termes, l'échantillon est balayé, point par point, par un laser focalisé dans le but de produire les différentes profondeurs de l'image. Les fluorophores placés à certains points de l'échantillon sont alors excités par le faisceau laser, produisant une fluorescence. Cette lumière passe, ensuite, dans le sténopé, permettant l'élimination des rayons lumineux hors-focus, et est détectée par le capteur; l'image est, ainsi, créée.

Les couleurs des images dans la microscopie LSCM par fluorescence proviennent des fluorophores utilisés pour marquer les points de l'échantillon. Ces fluorophores correspondent à des protéines fluorescentes qui ont deux longueurs d'ondes importantes : la longueur d'onde d'absorption et la longueur d'onde d'émission. Les protéines, lorsqu'elles absorbent la lumière, émettent une lumière à une longueur d'onde supérieure à celle absorbée. De ce fait, lorsque l'échantillon est balayé par le laser, les fluorophores excités retransmettre une lumière à une certaine longueur d'onde. Celle-ci est, ensuite, captée par le détecteur et crée ainsi la gamme de couleurs retrouvées \textcolor{red}{(Source 2)}. Comme ces protéines sont fabriquées spécialement pour se joindre à des parties spécifiques des cellules, la fluorescence permet l'observation ciblée de différents éléments de la matière observée, à l'aide de canaux différents.

En ce qui a trait aux différents canaux de l'image, le premier canal correspond aux cellules cancéreuses de l'échantillon, soit les parties anormales à détecter. Le deuxième canal correspond au cytoplasme, soit la partie gélatineuse du contenu de la cellule. Celle-ci se trouve entre la membrane plasmique et le noyau \textcolor{red}{(Source 3)}. Finalement, le dernier canal de l'image correspond au \textit{bright field}. Ce canal représente une image en lumière blanche traditionnelle, la lumière est transmise à travers l'échantillon et le contraste est généré par l'absorption de celle-ci dans les zones denses. Donc, ce canal permet d'offrir une visualition générale de la cellule \textcolor{red}{(Source 4)}.

%tu peux tag certaines parties d'organismes vivants avec des fluorophores.
%les couleurs viennent de ces différents fluorophores un canal analyse qu'un type de fluorophore

% Source 1 : https://www.ncbi.nlm.nih.gov/pmc/articles/PMC6961134/

% Source 2 : Protéines-fluorescentes-cours

% Source 3 : https://www.futura-sciences.com/sante/definitions/biologie-cytoplasme-125/

% Source 4 : https://www.sciencedirect.com/topics/medicine-and-dentistry/bright-field-microscopy

\subsection{Question 2}
Pour trouver une estimation de la résolution du système, la taille réelle d'une cellule cancéreuse de prostate est d'abord trouvée. De cette manière, le diamètre d'une cellule réelle est d'environ 16.6$\pm$2.9 $\mu$m avec une épaisseur d'environ 15.1$\pm$2.6 $\mu$m  \textcolor{red}{(Source A)}. Ensuite, à partir de l'image 3D donnée, le nombre de pixel pour le diamètre de de la cellule est évaluée. Donc, pour la cellule sur la figure \ref{3d_cells}, un diamètre d'environ 25$\pm$1 pixels et une épaisseur de 12$\pm$1 sont comptés. Avec ces données, il est possible de calculer la résolution latérale et axiale. Donc, la résolution est donnée par l'équation suivante :
\begin{equation}
  R=\frac{\text{Diamètre de cellule réelle}}{\text{Nb de pixels}}
\end{equation}
La résolution latérale correspond au plan \textit{xy}, ainsi le résultat de celle-ci est de :
\begin{equation}
  R_{L}=\frac{16.6\,\mu\mathrm{m}}{25\,\mathrm{Pixels}}=0.66\,\frac{\mu\mathrm{m}}{\mathrm{Pixels}}
\end{equation}
En utilisant la même équation pour la résolution axiale, correspondant au plan \textit{z}, le résultat obtenu est de :
\begin{equation}
  R_{A}=\frac{15.1\,\mu\mathrm{m}}{12\,\mathrm{Pixels}}=1.3\,\frac{\mu\mathrm{m}}{\mathrm{Pixels}}
\end{equation}
Pour les deux valeurs obtenues, les incertitudes sont calculées à l'aide de la formule suivante :
\begin{equation}
  \Delta R=\sqrt{\left(\frac{\Delta V_{R}}{V_{R}}\right)^{2}+\left(\frac{\Delta V_{P}}{V_{P}}\right)^{2}}
\end{equation}
Où $\Delta V_{R}$ est l'incertitude de la valeur réelle de la cellule, $V_{R}$ est la valeur réelle de la cellule, $\Delta V_{P}$ est l'incertitude de la valeur comptée, et $V_{P}$ est la valeur comptée. Ainsi, les valeurs estimées de résolution avec leur incertitude sont les suivantes :
\begin{align*}
  R_{L}&=0.66\pm0.2\,\mu\mathrm{m} & R_{A}&=1.3\pm0.2\,\mu\mathrm{m} \\
\end{align*}


%TODO : trouver la source, puis confirmer apres que ça fit avec l'image 3d

% Source A : https://www.ncbi.nlm.nih.gov/pmc/articles/PMC5578258/

\subsection{Question 3}
Le rôle d'un sténopé dans un système optique est d'éliminer les faisceaux lumineux ne provenant pas du plan focal, permettant d'améliorer la définition et la netteté des images. Cependant, la taille de ce diaphrame est important dans le balancement de la résolution et de la quantité de lumière admise par le système. En effet, si la taille du sténopé est trop petite, la quantité de lumière provenant du plan focal lui-même est réduite, engendrant une perte de puissance. De plus, puisque la quantité de lumière atteignant le détecteur est réduite, le rapport entre le bruit et les signaux diminue, ce qui rend le bruit de fond plus apparent. De cette manière, les images peuvent apparaître moins définies et plus sombres \textcolor{red}{(Source 5)}. 

Par exemple, pour un système parfait utilisant un profil de faisceau gaussien et opérant à une longueur d'onde de 600 nm, la taille du sténopé le plus petit qu'il serait possible d'utiliser peut être calculée à l'aide de la formule suivante \textcolor{red}{(Source X)} :
\begin{equation}
  \theta=\frac{\lambda}{n\pi\omega_{0}}
\end{equation}
Où $\lambda$ est la longueur d'onde, soit de 600 nm, $\omega_{0}$ est la taille du faisceau, et $n$ est l'indice de réfraction. La taille du sténopé peut être estimée à partir de la taille du faisceau. En effet, dans un système parfait, la taille du faisceau est approximativement proportionnelle à celle du sténopé. Donc, pour trouver la valeur de $\omega_0$, il faut estimer la valeur de l'angle de divergence du faisceau. Pour trouver cet angle, l'angle de convergence de la dernière lentille est calculé.
\begin{equation}
  \theta=2\tan\left(\frac{\phi/2}{f}\right)
\end{equation}
En utilisant les valeurs numériques données dans le procédurier, un angle de 0.7 radians est trouvé. Ainsi, en isolant la valeur recherchée dans l'équation 5 et en calculant numériquement, la taille la plus petite pour le sténopé est de :
\begin{equation}
  \omega_{0}=\frac{\lambda}{n\pi\theta}=272.84\,\mathrm{nm}
\end{equation}
% Source 5 : Confocal Microscopy 

% Source X : https://en.wikipedia.org/wiki/Gaussian_beam

\subsection{Question 4}
La résolution latérale correspond à la capacité d'un système à distinguer les points d'un objet qui sont situés à proximité l'un de l'autres alors que le contraste, étroitement lié à la résolution, est défini comme la quantité de photons collectés à partir de l'échantillon \textcolor{red}{(Source 6)}. Les systèmes confocaux permettent d'améliorer ces deux paramètres avec l'ajout d'un diaphrame optique. Comme mentionné, l'ajout de ce sténopé permet de limiter les faisceaux lumineux, ce qui élimine les rayons en provenance des plans non-focaux. Puisque la lumière atteignant le détecteur ne provent que du plan focal de l'échantillon, le bruit créé par les rayons extérieurs est réduit. De cette manière, le flou créé par ces rayons est éliminé et la différence de lumiosité entre les structures est mieux définie, offrant ainsi une amélioration dans la résolution et dans le contraste \textcolor{red}{(Source 7)}.

Par exemple, dans la figure \ref{3d_cells} utilisant le système confocal, il est possible d'observer, en ce qui a trait au contraste, que des différences de lumiosité entre les canaux. En effet, les détails des divers canaux, soient le rouge, le bleu et le vert, peuvent être distingués clairement les uns des autres. Cependant, dans le cas de celui qui n'utilise pas le système confocal, soit le suivant :
\begin{figure}[H]
  \centering
  \includegraphics[scale=0.2]{xy_plane_non_confocal.png}
  \caption{Image de la cellule sans le système confocal.}
  \label{non-confocal}
\end{figure}
Les détails sont légèrement moins marqués que pour ceux de la figure \ref{3d_cells}, particulièrement dans les zones où les couleurs se chevauchent. De plus, pour l'image confocale, celle est mieux définie avec des contours plus nets autour des structures, facilitant la distinction des composantes. Ainsi, avec ces exemples, il est possible de visualiser l'augmentation de la résolution et du contraste lorsqu'un système confocal est utilisé.

Bien que les systèmes confocaux font preuve de plusieurs avantages, ceux-ci comportent quelques désavantages. Un premier désagrément est la vitesse d'acquisition des données. Puisque les points sont balayés individuellement, l'acquisition des données est trop lente pour obtenir des informations sur des systèmes biologiques rapide \textcolor{red}{(Source 8)}. De plus, les microscopes confocaux n'utilisent qu'un seul photon pour exciter les fluorophores, ce qui peut résulter à plus grande dispersion de la lumière dans l'image que, par exemple, des sytèmes multiphotoniques \textcolor{red}{(Source 9)}.

% Source 6 : https://www.olympus-lifescience.com/en/microscope-resource/primer/techniques/confocal/resolutionintro/

% Source 7 : https://www.leica-microsystems.com/science-lab/life-science/pinhole-effect-in-confocal-microscopes/#:~:text=The%20lateral%20resolution%20at%201,30%25%20at%20pinhole%20diameter%20zero.

% Source 8 : https://www.tandfonline.com/doi/full/10.2144/000112089#:~:text=A%20key%20limitation%20in%20the,information%20about%20rapid%20biological%20processes.

% Source 9 : https://www.edmundoptics.ca/knowledge-center/application-notes/microscopy/confocal-microscopy/#:~:text=Compared%20to%20multiphoton%20microscopy%2C%20confocal,as%20deep%20into%20a%20sample.

\section{Conclusion}

%TODO

\begin{lstlisting}[language=python]
print("l'exam d'optique etait deg")
\end{lstlisting}



\clearpage

%TODO
% \bibliographystyle{unsrtnat}
% \bibliography{My_Library}

\end{document}
