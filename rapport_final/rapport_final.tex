\documentclass[11pt,letterpaper]{article}
\usepackage[top=3cm, bottom=2cm, left=2cm, right=2cm, columnsep=20pt]{geometry}
\usepackage{pdfpages}
\usepackage{graphicx}
\usepackage{etoolbox}
\apptocmd{\sloppy}{\hbadness 10000\relax}{}{}
% \usepackage[numbers]{natbib}
\usepackage[T1]{fontenc}
\usepackage{ragged2e}
\usepackage[french]{babel}
\usepackage{listings}
\usepackage{color}
\usepackage{soul}
\usepackage[utf8]{inputenc}
\usepackage[export]{adjustbox}
\usepackage{caption}
\usepackage{amsmath}
\usepackage{amssymb}
\usepackage{float}
\usepackage{csquotes}
\usepackage{fancyhdr}
\usepackage{wallpaper}
\usepackage{siunitx}
\usepackage[indent]{parskip}
\usepackage{textcomp}
\usepackage{gensymb}
\usepackage{multirow}
\usepackage[hidelinks]{hyperref}
\usepackage{abstract}
\renewcommand{\abstractnamefont}{\normalfont\bfseries}
\renewcommand{\abstracttextfont}{\normalfont\itshape}
\usepackage{titlesec}
\titleformat{\section}{\large\bfseries}{\thesection}{1em}{}
\titleformat{\subsection}{\normalsize\bfseries}{\thesubsection}{1em}{}
\titleformat{\subsubsection}{\normalsize\bfseries}{\thesubsubsection}{1em}{}

\usepackage{xcolor}
\definecolor{codegreen}{rgb}{0,0.6,0}
\definecolor{codegray}{rgb}{0.5,0.5,0.5}
\definecolor{codepurple}{rgb}{0.58,0,0.82}
\definecolor{backcolour}{rgb}{0.95,0.95,0.92}
\lstdefinestyle{mystyle}{
    backgroundcolor=\color{backcolour},   
    commentstyle=\color{codegreen},
    keywordstyle=\color{magenta},
    numberstyle=\tiny\color{codegray},
    stringstyle=\color{codepurple},
    basicstyle=\ttfamily\footnotesize,
    breakatwhitespace=false,         
    breaklines=true,                 
    captionpos=b,                    
    keepspaces=true,                 
    numbers=left,                    
    numbersep=5pt,                  
    showspaces=false,                
    showstringspaces=false,
    showtabs=false,                  
    tabsize=2
}
\lstset{style=mystyle}

\usepackage[most]{tcolorbox}
\newtcolorbox{note}[1][]{
  enhanced jigsaw,
  borderline west={2pt}{0pt}{black},
  sharp corners,
  boxrule=0pt, 
  fonttitle={\large\bfseries},
  coltitle={black},
  title={Note:\ },
  attach title to upper,
  #1
}

%----------------------------------------------------

\setlength{\parindent}{0pt}
\DeclareCaptionLabelFormat{mycaptionlabel}{#1 #2}
\captionsetup[figure]{labelsep=colon}
\captionsetup{labelformat=mycaptionlabel}
\captionsetup[figure]{name={Figure }}
\newcommand{\inlinecode}{\normalfont\texttt}
\usepackage{enumitem}
\setlist[itemize]{label=\textbullet}

\begin{document}
\begin{titlepage}
\center

\begin{figure}
    \ThisULCornerWallPaper{.4}{Polytechnique_signature-RGB-gauche_FR.png}
\end{figure}
\vspace*{2 cm}

\textsc{\Large \textbf{PHS2223 --} Introduction à l'optique moderne}\\[0.5cm]
\large{\textbf{Équipe : 04}}\\[1.5cm]

\rule{\linewidth}{0.5mm} \\[0.5cm]
\Large{\textbf{Expérience 1}} \\[0.2cm]
\text{Microscopie confocale}\\
\rule{\linewidth}{0.2mm} \\[2.3cm]

\large{\textbf{Présenté à}\\
  Guillaume Sheehy\\
  Esmat Zamani\\[2.5cm]
  \textbf{Par :}\\
  Émile \textbf{Guertin-Picard} (2208363)\\
  Laura-Li \textbf{Gilbert} (2204234)\\
  Tom \textbf{Dessauvages} (2133573)\\[3cm]}

\large{\today\\
Département de Génie Physique\\
Polytechnique Montréal\\}

\end{titlepage}

%----------------------------------------------------

\tableofcontents
\pagenumbering{roman}
\newpage

\pagestyle{fancy}
\setlength{\headheight}{14pt}
\renewcommand{\headrulewidth}{0pt}
\fancyfoot[R]{\thepage}

\pagestyle{fancy}
\fancyhf{}
\renewcommand{\headrulewidth}{1pt}
\fancyhead[L]{\textbf{PHS2223}}
\fancyhead[C]{titre}
\fancyhead[R]{date}
\fancyfoot[R]{\thepage}

\pagenumbering{arabic}
\setcounter{page}{1}

%----------------------------------------------------

\section{Résultats}

\subsection{Estimation de la résolution}

%TODO : graph avec sténopé

%TODO : graph sans sténopé

%TODO : justifier le choix de la fonction de fit

%TODO : estimer la resolution avec la largeur a mi-hauteur

\subsection{Analyse de données de LSCM en fluorescence}

%TODO : comparaison d'images

%TODO : volume de quelques cellules

\section{Discussion}

\subsection{Retour sur l'hypothèse}

%TODO : comparaision mesure axiale entre hypothese et expérience

\subsection{Analyse des causes d'erreurs}

%TODO

vieux laser avec deux pics d'intensité
mauvais alignement de base (surtout l'alignement des back reflections)
désalignement au fur et à mesure des manipulations
incertitudes (surtout sur le power meter)

\subsection{Question 1}

%TODO

(de ce que j'ai compris apres avoir jasé avec Noémie lol)
tu peux tag certaines parties d'organismes vivants avec des fluorophores.
les couleurs viennent de ces différents fluorophores
un canal analyse qu'un type de fluorophore

\subsection{Question 2}

%TODO : trouver la source, puis confirmer apres que ça fit avec l'image 3d

\subsection{Question 3}

%TODO

le pinhole améliore la netteté des images, la définition des cellules, etc.
perte de puissance avec un pinhole, un trop petit bloquerait toute image
voir sources pour le shit de plus petit sténopé possible  

\subsection{Question 4}

%TODO

faire reference a la figure prostate side by side

\section{Conclusion}

%TODO

\clearpage

%TODO
% \bibliographystyle{unsrtnat}
% \bibliography{My_Library}

\end{document}
